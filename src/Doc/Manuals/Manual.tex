


%***********************************************
%************ Dokumentklasse *******************
%***********************************************
%\documentclass[makeidx,sumlimits,twoside,a4paper,12pt]{report}
\documentclass[makeidx,sumlimits,a4paper,12pt]{report}


%***********************************************
%************ Geladene Pakete  *****************
%***********************************************
\usepackage[ansinew]{inputenc}     	%% Ansi
\usepackage{color}	 	   	%% Farben
\usepackage{graphicx}		%% Grafikpaket
\usepackage{calc}		   	%% Calc-Paket
\usepackage{layout}			%% Layout-Paket
\usepackage[bf]{caption2}
\usepackage{times}
\usepackage[pdftex, pdfmark, dvips, ps2pdf]{thumbpdf}

%***********************************************
%************ Eigene Befehle *******************
%***********************************************
\graphicspath{{pics/}{.}}	   			%% relative Pfadangabe f�r Grafiken und Bilder




%************ Textbox ***************************
\newcommand{\flexXYbox}[4]{\fbox{\rule[#1 cm]{0cm}{#2 cm}\makebox[#3 cm]{#4}}}


%***********************************************
%************* Grundeinstellungen **************
%***********************************************
\parindent0cm
\sloppy
\nonfrenchspacing
\makeindex				% tell \index to actually write the .idx file
\linespread{1.3}
	

%***********************************************
%************ Anfang des Dokumentes ************
%***********************************************
\begin{document}


%***********************************************
\pagenumbering{Roman}



%Inhaltsverzeichnis
\tableofcontents
\newpage
\pagenumbering{arabic}


%***********************************************
%chapter 1: Introduction
\chapter{Introduction}

This Manual will give you an overview of FreeCAD from the
users point of view. Although FreeCAD is a RAD (Rapid Application Development)
Framework, it's also an Application for End users in the Field of 
mechanical engineering and similar usage.

In the early stages of the FreeCAD development you can not really do a
lot useful things with it. But I think after a while it becomes more
and more. Therefore the Manual is at the moment not very Big, but it
will become more as the functionality of FreeCAD grows. 


J�rgen Riegel 2002


%***********************************************
%Chapter 2: Installation
\chapter{Installation}

\section{On Unixes}

\subsection{Environment}


\section{On Windows}

\subsection{Register as Server}



%***********************************************
%Chapter 2: Basics
\chapter{Basic Tutorial}


\section {Basics}

\subsection {The user interface}

\subsection {Documents}

\subsection {Macro recording}


\section {Workbenches}

\subsection {Part}

\subsection {Test}


%***********************************************
%Chapter 2: Server
\chapter{Server}

\section {(D)COM server}

\section {Web services}

\section {RPC}

%***********************************************
%Chapter 2: Appendix
\chapter{Appendix}

\section {Command line arguments}

\section {Contribution}

Thanks to Werner Mayer for his help on various things (like picture conversion)

Thanks goes also to Trolltech for releasing the great QT Toolkit for Open Source 
development.

Thanks to Matra Datavision which makes this project possible in the first place by
releasing CasCade as Open Source.

Thanks Guido van Rossum for Python.



%***********************************************
%************ Ende des Dokumentes **************
%***********************************************
\end{document}
